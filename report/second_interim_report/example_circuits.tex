%% LyX 2.2.4 created this file.  For more info, see http://www.lyx.org/.
%% Do not edit unless you really know what you are doing.
\documentclass[english]{article}
\usepackage{lmodern}
\usepackage[T1]{fontenc}
\usepackage[latin9]{inputenc}
\usepackage{geometry}
\geometry{verbose,tmargin=3cm,bmargin=3cm,lmargin=2.5cm,rmargin=2.5cm}
\usepackage{textcomp}
\usepackage{amsmath, amstext}
\usepackage{graphicx, float}
\usepackage{pdfpages}
\usepackage{booktabs}
\usepackage{multirow}
\usepackage{pdfpages}
\usepackage{fancyvrb}
\usepackage{pdflscape}
\usepackage{listings,multicol}



\begin{document}

\subsection*{Example Circuit A --- Simple Circuit}

\vspace{-0.4cm}
\begin{figure}[H]
    \begin{center}
    \includegraphics[width = 0.7\textwidth]{Simple Circuit.png}
    \vspace{-1cm}
    \caption{XOR Gate}
    \end{center}
    \end{figure}

\begin{Verbatim}[numbers=left,xleftmargin=5mm]
    #Example circuit - Simple circuit (As per first interim report);
    #XOR gate;
    
    devices(
        a is OR;
        b is NAND;
        c is AND;
        sw1, sw2 are SWITCH;
    )
    
    initialise(
        sw1, sw2 are HIGH;
        a, b, c have 2 inputs;
    )
    
    connections(
        a(
            sw1 is connected to a.I1;
            sw2 is connected to a.I2;
        )
    
        b(  
            sw1 is connected to b.I1;
            sw2 is connected to b.I2;
        )
    
        c(  
            a is connected to c.I1;
            b is connected to c.I2;
        )
    )
    
    monitors(
        c;
    )
\end{Verbatim}


\newpage
\subsection*{Example Circuit B --- Complex Circuit}

\vspace{-0.4cm}
\begin{figure}[H]
    \begin{center}
    \includegraphics[width = \textwidth]{Complex Circuit.png}
    \caption{2 Bit Counter with Outputs XOR}
    \end{center}

    \end{figure}

\begin{Verbatim}[numbers=left,xleftmargin=5mm]
    #Example circuit - Complex circuit (As per first interim report);
    #2 bit counter with outputs XOR;
    
    devices(
        a is NOR;
        b, c, d, e are NAND;
        f is DTYPE;
        g is XOR;
        sw1, sw2 are SWITCH;
        clk1 is CLOCK;
    )
    
    initialise(
        a, b, c, d, e, g have 2 inputs;
        sw1, sw2 are LOW;
        clk1 cycle length 5;
    )
    
    connections(
        a(
            e to a.I1;
            e to a.I2;
        )
    
        b(  
            e to b.I1;
            clk1 to b.I2;
        )
    
        c(  
            clk1 to c.I1;
            a to c.I2;
        )
    
        d(
            b to d.I1;
            e to d.I2;
        )
    
        e(  
            d to e.I1;
            c to e.I2;
        )
    
        f(
            f.QBAR to f.DATA;
            d to f.CLK;
            sw1 to f.SET;
            sw2 to f.CLEAR;
        )
    
        g(  
            d to g.I1;
            f.Q to g.I2;
        )
    )
    
    monitors(
        d, f.Q, g;
    )
\end{Verbatim}

\newpage
\subsection*{Example Circuit C --- 50 Switches}

\vspace{-0.4cm}
\begin{figure}[H]
    \begin{center}
    \includegraphics[width = \textwidth]{50 Switches.png}

    \caption{50 Switches}
    \end{center}
    \end{figure}

\begin{Verbatim}[numbers=left,xleftmargin=5mm]
    #Example circuit - 50 switches;

    devices(
        sw1, sw2, sw3, sw4, sw5, sw6, sw7, sw8, sw9, sw10,
        sw11, sw12, sw13, sw14, sw15, sw16, sw17, sw18, sw19, sw20,
        sw21, sw22, sw23, sw24, sw25, sw26, sw27, sw28, sw29, sw30,
        sw31, sw32, sw33, sw34, sw35, sw36, sw37, sw38, sw39, sw40,
        sw41, sw42, sw43, sw44, sw45, sw46, sw47, sw48, sw49, sw50 are SWITCH;
    )
    
    initialise(
        sw1, sw2, sw3, sw4, sw5, sw6, sw7, sw8, sw9, sw10,
        sw11, sw12, sw13, sw14, sw15, sw16, sw17, sw18, sw19, sw20,
        sw21, sw22, sw23, sw24, sw25 are LOW;
        sw26, sw27, sw28, sw29, sw30, sw31, sw32, sw33, sw34, sw35,
        sw36, sw37, sw38, sw39, sw40, sw41, sw42, sw43, sw44, sw45,
        sw46, sw47, sw48, sw49, sw50 are HIGH;
    )
    
    connections(
    )
    
    monitors(
        sw1, sw2, sw3, sw4, sw5, sw6, sw7, sw8, sw9, sw10,
        sw11, sw12, sw13, sw14, sw15, sw16, sw17, sw18, sw19, sw20,
        sw21, sw22, sw23, sw24, sw25, sw26, sw27, sw28, sw29, sw30,
        sw31, sw32, sw33, sw34, sw35, sw36, sw37, sw38, sw39, sw40,
        sw41, sw42, sw43, sw44, sw45, sw46, sw47, sw48, sw49, sw50;
    )
    
\end{Verbatim}

\newpage
\subsection*{Example Circuit D --- SR Bistable}

\vspace{-0.4cm}
\begin{figure}[H]
    \begin{center}
    \includegraphics[width = 0.65\textwidth]{SR Bistable.png}

    \caption{SR Bistable}
    \end{center}
    \end{figure}

\begin{Verbatim}[numbers=left,xleftmargin=5mm]
    #Example circuit - SR bistable;

    devices(
        a, b, c, d are NAND;
        sw1, sw2 are SWITCH;
        clk1 is CLOCK;
    )
    
    initialise(
        sw1, sw2 are LOW;
        a, b, c, d have 2 inputs;
        clk1 cycle 5;
    )
    
    connections(
        a(
            sw1 to a.I1;
            clk1 to a.I2;
        )
    
        b(  
            clk1 to b.I1;
            sw2 to b.I2;
        )
    
        c(  
            a to c.I1;
            d to c.I2;
        )
    
        d(  
            c to d.I1;
            b to d.I2;
        )
    )
    
    monitors(
        c, d;
    )
\end{Verbatim}

\newpage
\subsection*{Example Circuit E --- Divide by 3 Circuit}

\vspace{-0.4cm}
\begin{figure}[H]
    \begin{center}
    \includegraphics[width = 0.65\textwidth]{Divide by 3.png}

    \caption{Divide by 3 Circuit}
    \end{center}
    \end{figure}

\begin{Verbatim}[numbers=left,xleftmargin=5mm]
    #Example circuit - divide (clock frequency) by 3;

    devices(
        a is AND;
        b, c are DTYPE;
        sw1, sw2, sw3, sw4 are SWITCH;
        clk1 is CLOCK;
    )
    
    initialise(
        a has 2 inputs;
        sw1, sw2, sw3, sw4 are LOW;
        clk1 cycle 9;
    )
    
    connections(
        a(
            b.QBAR to a.I1;
            c.QBAR to a.I2;
        )
    
        b(  
            a to b.DATA;
            clk1 to b.CLK;
            sw1 to b.SET;
            sw2 to b.CLEAR;
        )
    
        c(  
            b.Q to c.DATA;
            clk1 to c.CLK;
            sw3 to c.SET;
            sw4 to c.CLEAR;
        )
    )
    
    monitors(
        c.Q;
    )
\end{Verbatim}

    
\newpage
\subsection*{Example Circuit F --- Blank Circuit}
\hfill\\


\begin{Verbatim}[numbers=left,xleftmargin=5mm]
    #Example circuit - Blank circuit;

    devices(
    )
    
    initialise(
    )
    
    connections(
    )
    
    monitors(
    )
\end{Verbatim}
\hfill\\
Alternatively, a blank .txt file will also suffice

\end{document}