%% LyX 2.2.4 created this file.  For more info, see http://www.lyx.org/.
%% Do not edit unless you really know what you are doing.
\documentclass[english]{article}
\usepackage{lmodern}
\usepackage[T1]{fontenc}
\usepackage[latin9]{inputenc}
\usepackage{geometry}
\geometry{verbose,tmargin=3cm,bmargin=3cm,lmargin=2.5cm,rmargin=2.5cm}
\usepackage{textcomp}
\usepackage{amsmath, amstext}
\usepackage{graphicx, float}
\usepackage{pdfpages}
\usepackage{booktabs}
\usepackage{multirow}
\makeatletter

%%%%%%%%%%%%%%%%%%%%%%%%%%%%%% LyX specific LaTeX commands.
%% Because html converters don't know tabularnewline
\providecommand{\tabularnewline}{\\}

\makeatother

\usepackage{babel}
\begin{document}

% \title{GF2: Software\\
% First Interim Report}

% \author{Yi Chen Hock - ych31}
% \maketitle
\begin{center}
{\bf ENGINEERING TRIPOS PART II A}
\end{center}
\vspace{0.5cm} {\bf GF2 \hfill SOFTWARE}
\vspace{0.5cm}
\begin{center}
{\bf LOGIC SIMULATOR \\\hfill
\\{\em First Interim Report}\\\hfill
\\Name: Yi Chen Hock, Michael Stevens, Cindy Wu
\\College: Queens', Robinson, John's\\\hfill
\\
}
\end{center}
\rule{15.7cm}{0.5mm}

\vspace{0.5cm}
\tableofcontents

\newpage
\section{Introduction}
In this lab we will investigate a semiconductor laser. We will generate light-current and voltage-current curves for the laser and measure the optical spectrum output from the laser in order to understand how these properties relate to each other and the properties of the laser. 

\section{Part I}

% \subsection{L-I and V-I plots of CD laser with errors}
% \begin{figure}[H]
% \begin{centering}
% \includegraphics[width=0.6\paperwidth]{imgs/graph.png}
% \par\end{centering}
% \caption{Plot of V-I and L-I for CD Laser\label{fig:graph} }
% \end{figure}

Figure \ref{fig:graph} shows the light-current and voltage-current relationships for the laser (see Appendix \ref{sec:appendix1} for the tabulated raw data). We can see from the plot that the photodiode output follows the characteristic shape of a laser: no current below approximately 1.5V and linearly increasing with current after a threshold current of $I_{\textrm {th}} = 44$mA. The voltage output saturates when the current reaches 84mA, so the last three data points (in the original table) have been neglected in the plot. 

The current values were calculated assuming a $20\Omega$ resistor between the current monitor (CH2) and ground and the Voltage was calculated from the difference between CH1 and CH2. 

The errors were estimated by taking half of the thickness of the lines seen in the oscilloscope output (figure \ref{fig:scope-img}). The errors in CH1, CH2 and CH3 are $\pm 0.1$V, $\pm 0.05$V and $\pm 0.06$V respectively, so the error in voltage (CH1-CH2) was calculated from the sum of the errors in CH1 and CH2. The error of the current ($\pm 2.5$mA) is calculated by the error in CH2 divided by 20 and multiplied by 1000 - these error bars were neglected in the above plot for clarity. 

% \begin{figure}[H]
% \begin{centering}
% \includegraphics[width=0.3\paperwidth]{imgs/scope_image.jpg}
% \par\end{centering}
% \caption{Oscilloscope output of laser at saturation\label{fig:scope-img} }
% \end{figure}

\subsection{Explanation of the different regimes of operation of the device}

Below the bandgap, only LED action occurs in the diode - very little light is emitted as there are only a few electrons in the conduction bad as the photons are just generated by spontaneous emission. This current is obscured by the noisy reading it is much smaller than the photodiode output during lasing. 

At or above the bandgap voltage, the diode experiences lasing action, where stimulated emission is much more likely than spontaneous absorption. Near the diode junction, population inversion occurs (more free carriers in conduction band than valence band), so the laser light output rapidly increases. 

\subsection{Estimation of stray series resistance and lasing wavelength of the device}

The line of best fit for voltage in Figure \ref{fig:graph} are fitted using least squares and their equations can be obtained using the LINEST function in Excel: $V=3I+1.57$, where I is given in Amperes. If we model the diode as a constant voltage drop with some stray resistance, $V=V_d+IR_s$, we can approximate the voltage drop $V_d$ as 1.57V and stray resistance $R_s$ as 3$\Omega$.

The LINEST function also returns the errors for the slope and intercept (calculated by the 95\% confidence interval). This gives an uncertainty of $\pm 0.09$V in $V_d$ and $\pm 1\Omega$ in $R_s$. There is a large percentage uncertainty of 33\% in the value of stray resistance. 

We can now use the value of $V_d$ to estimate the mean wavelength of the laser, assuming that this is the average potential difference between electrons in the conduction and valence bands: 

$$E_{ph} = \frac{hc}{\lambda_0}=eV_d$$
$$\implies \lambda_0 = \frac{hc}{eV_d}=\frac{6.63\times 10^{-34}\times 3\times 10^8}{1.6\times 10^{-19}\times 1.57} = 792 \textrm {nm}$$

As h, c and e are all fundamental constants, the percentage uncertainty in $\lambda_0$ is the same as $V_d$ at around 6\%, so its uncertainty is $\pm 48$nm. We should note that the measurements obtained optically (from the light output) were a lot less accurate and precise than the electronic measurements where the measurement probes would be directly connected to the circuit. There are a number of different factors that may have affected the accuracy of the optical photodiode measurements, such as varying ambient light levels and the coupling efficiency between the laser and the photodiode. As a result, these measurements should just be used to observe general trends in the relationships between the different parameters investigated and would not be suitable for performing calculations to a high degree of accuracy. 

\subsection{Effect of temperature on obtained measurements}

The relationship between threshold current and temperature is given by: 

$$I_{\textrm th} = I_0 \exp \left(\frac{T}{T_0}\right)$$

We can see that temperature has a significant effect on p-n junction recombination. It changes the number of thermally excited electrons within the conduction band, so the characteristic time for an electron to fall from the conduction band to the valence band reduces with increasing temperature. This means more current is required to maintain the same amount of population inversion for lasing. Typically $T_0=150K$ for GaAs lasers. Figure \ref{fig:temp} shows the effect of temperature on the light output of lasers and LEDs. 

% \begin{figure}[H]
% \begin{centering}
% \includegraphics[width=0.6\paperwidth]{imgs/temp.png}
% \par\end{centering}
% \caption{Relationship between temperature and light output for a laser (left) and LED (right)\label{fig:temp} }
% \end{figure}

During the experiment, the temperature remained at $21^\circ$C throughout, so temperature is not likely to have had a significant effect on the measurements that were obtained. 

\section{Part II}

\subsection{Plots of obtained optical spectra}

% \begin{figure}[H]
% \begin{centering}
% \includegraphics[width=0.3\paperwidth]{imgs/narrow14.jpg}\hspace{1cm}\includegraphics[width=0.3\paperwidth]{imgs/narrow17.jpg}
% \par\end{centering}
% \caption{Spectra for 14mA
% (left) and 17mA (right) using narrow wavelength span (5nm)\label{fig:narrow} }
% \end{figure}

% \begin{figure}[H]
% \begin{centering}
% \includegraphics[width=0.3\paperwidth]{imgs/large14.jpg}\hspace{1cm}\includegraphics[width=0.3\paperwidth]{imgs/large17.jpg}
% \par\end{centering}
% \caption{Spectra for 14mA
% (left) and 17mA (right) using large wavelength span (500nm, 100nm)\label{fig:large} }
% \end{figure}

\subsection{Recorded data on centre wavelength $\lambda_0$, peak output power $P_0$ and 3dB spectral width}

\begin{table}[H]
\begin{center}
\begin{tabular}{@{}l|ll@{}}
\textbf{}               & \textbf{Below threshold (14mA)} & \textbf{Above threshold (17mA)} \\ \midrule
\textbf{$\lambda_0$ ($\mu$m)}      & 1.558                           & 1.557                          \\
\textbf{$P_0$ (dBm)}        & -73.6                           & -30.0                             \\
\textbf{3dB width (nm)} & 24                              & 0.06                           
\end{tabular}
\par\end{center}
\caption{Data on spectral characteristics above and below threshold current}
\end{table}

\subsection{Explanation of the observed differences in recorded spectra and their relation to the fundamental light emission process in semiconductor devices}

The centre wavelength ($\lambda_0=\frac{hc}{E_g}$) remains unchanged for the two currents because it is determined by the bandgap of the junction material and not dependent on the applied current. 

The difference in maximum power in the spectra is 43.7dBm = 23,400. This implies the diode outputs this much more power above threshold than below threshold. This is expected as there is lasing action (population inversion and stimulated emission near the junction) above threshold, leading to much greater power output. 

Below the threshold, the 3dB width is much greater - across the frequency and wavelengths of the distribution is more spread out as mostly thermally excited electrons are being emitted, leading to a variation in energy output. 

\subsection{Measured mode spacing above and below threshold}

The mode spacing was measured across multiple peaks and divided to give a more accurate value. Below threshold the mode spacing is 1.19nm. Above threshold it is also 1.19nm. 

\subsection{Calculation of refractive index in semiconductor device}

Across the laser cavity, Fabry-Perot modes occur from the standing waves: 
\begin{equation*}
\frac{m}{2}\lambda_g=L
\qquad
\lambda_g=\frac{\lambda}{n}
\qquad
c=f\lambda
\end{equation*}

$$\implies f=\frac{mc}{2nL}$$
$$\triangle f = f_{m+1}-f_m=\frac{c}{2nL}$$
$$\triangle \lambda = \frac{\lambda^2}{2nL}$$

\noindent As we know the length of laser box was $300\mu$m. Using centre wavelength of 1.558$\mu$m, we can calculate the value of $n$: 

$$n=\frac{\lambda^2}{2\triangle \lambda L} = \frac{(1.558\times 10^{-6})^2}{2\times 1.19\times 10^{-9} \times 300\times 10^{-6}}=3.4$$

\section{Conclusion}
The semiconductor laser was investigated with two different methods in this lab. The results were used to characterise the diode behaviour and observe its properties. Care was taken to account for the uncertainties of various measurements obtained during the experiment. 

\section{Appendix}

\subsection{Tabulated results for CD laser\label{sec:appendix1}}

\begin{table}[H]
\begin{center}
\begin{tabular}{@{}lllll@{}}
\toprule
\textbf{\begin{tabular}[c]{@{}l@{}}Voltage\\ Monitor/V\\ (CH1)\end{tabular}} & \textbf{\begin{tabular}[c]{@{}l@{}}Current\\ Monitor/V\\ (CH2)\end{tabular}} & \textbf{\begin{tabular}[c]{@{}l@{}}Current/mA\\ (CH2/20$\Omega$)\end{tabular}} & \textbf{\begin{tabular}[c]{@{}l@{}}Voltage/V\\ (CH1-CH2)\end{tabular}} & \textbf{\begin{tabular}[c]{@{}l@{}}Photodiode\\ Output/V\\ (CH3)\end{tabular}} \\ \midrule
2.5                                                                          & 0.88                                                                         & 43.8                                                                   & 1.63                                                                   & 0.00                                                                           \\
2.6                                                                          & 1.00                                                                         & 50.0                                                                   & 1.60                                                                   & 0.13                                                                           \\
2.8                                                                          & 1.06                                                                         & 53.0                                                                   & 1.69                                                                   & 0.30                                                                           \\
2.9                                                                          & 1.13                                                                         & 56.3                                                                   & 1.73                                                                   & 0.45                                                                           \\
3.0                                                                          & 1.19                                                                         & 59.4                                                                   & 1.81                                                                   & 0.55                                                                           \\
3.1                                                                          & 1.25                                                                         & 62.5                                                                   & 1.85                                                                   & 0.70                                                                           \\
3.2                                                                          & 1.30                                                                         & 65.0                                                                   & 1.90                                                                   & 0.88                                                                           \\
3.2                                                                          & 1.38                                                                         & 68.8                                                                   & 1.83                                                                   & 1.00                                                                           \\
3.2                                                                          & 1.44                                                                         & 71.9                                                                   & 1.76                                                                   & 1.13                                                                           \\
3.3                                                                          & 1.50                                                                         & 75.0                                                                   & 1.80                                                                   & 1.24                                                                           \\
3.4                                                                          & 1.56                                                                         & 78.1                                                                   & 1.84                                                                   & 1.38                                                                           \\
3.5                                                                          & 1.63                                                                         & 81.3                                                                   & 1.88                                                                   & 1.50                                                                           \\
3.5                                                                          & 1.69                                                                         & 84.4                                                                   & 1.81                                                                   & 1.63                                                                           \\
3.6                                                                          & 1.75                                                                         & 87.5                                                                   & 1.85                                                                   & 1.63                                                                           \\
3.6                                                                          & 1.81                                                                         & 90.6                                                                   & 1.79                                                                   & 1.68                                                                           \\
3.6                                                                          & 1.88                                                                         & 93.8                                                                   & 1.73                                                                   & 1.68                                                                           \\ \bottomrule
\end{tabular}
\par\end{center}
\caption{Raw data for CD laser}
\end{table}

\end{document}
