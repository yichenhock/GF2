%% LyX 2.2.4 created this file.  For more info, see http://www.lyx.org/.
%% Do not edit unless you really know what you are doing.
\documentclass[english]{article}
\usepackage{lmodern}
\usepackage[T1]{fontenc}
\usepackage[latin9]{inputenc}
\usepackage{geometry}
\geometry{verbose,tmargin=2cm,bmargin=1.5cm,lmargin=2.5cm,rmargin=2.5cm}
\usepackage{textcomp}
\usepackage{amsmath, amstext}
\usepackage{graphicx, float}
\usepackage{pdfpages}
\usepackage{booktabs}
\usepackage{multirow}
\usepackage{pdfpages}
\usepackage{fancyvrb}
\usepackage{pdflscape}
\usepackage{listings,multicol}

\usepackage{enumitem,kantlipsum}
\usepackage{fancyhdr}

\pagestyle{fancy}
\fancyhf{}
\rhead{Yi Chen Hock / ych31}
\lhead{GF2 Software}
% \rfoot{something}

\begin{document}
\pagenumbering{gobble}
\section*{Logic Simulator: User Guide}

\vspace{-0.3cm}
\begin{figure}[H]
\begin{center}
\includegraphics[width = 0.9\textwidth]{gui_screenshot_labelled.png}
\vspace{-0.1cm}
\caption{Screenshot of the GUI\label{fig:gui}}
\end{center}
\end{figure}

\vspace{-0.3cm}

Logic Simulator is a tool that allows the user to read in, edit and simulate a logic circuit defined in a user-provided circuit definition file\footnote{Definition files follow the EBNF grammar defined in the first interim report.}. Figure \ref{fig:gui} shows a screenshot of the Graphical User Interface (GUI) labelled with the following features:

\begin{multicols}{3}
\begin{enumerate}[leftmargin=*]
    \item Menubar containing 3 options: 
    
    \textit{About}, \textit{Save As} and \textit{Quit}.
    \item Opens a definition file 
    \item Saves the current definition file
    \item Creates a new definition file
    \item Compiles the edited definition file in the \textit{Circuit Definition} tab for any errors and initialises the inputs and monitors 
    \item Runs the code for the specified number of simulation cycles from scratch. 
    \item Continues the simulation for the specified number of cycles.
    \item Resets the simulation.
    \item Allows the user to specify the number of simulation cycles.
    \item Saves the plot as an image.
    \item Displays the user guide.
    \item Quits the application.
    \item Signal trace plot.
    \item Side panel containing four tabs: 
    
    \textit{Output}: Console log. Text commands can also be run here.
    
    \textit{Circuit Definition}: Editable area for the loaded definition file.
    
    \textit{Inputs}: List of input switches. Has buttons which allow the switch states to be toggled ON or OFF.
    
    \textit{Monitors}: List of signals to be monitored. Allows components to be added or removed. 

    \item Statusbar.
    \item Path name of the current definition file. 
\end{enumerate}
\end{multicols}

\noindent A command-line interface is also available by running \verb+python logsim/logsim.py -c <pathname>+. Typing \verb+h+ will display a list of possible commands. Example definition files can be found in the \verb+./logsim/examples+ folder\footnote{See overleaf for the circuit diagrams of the available example definition files.}. For further information on how to install and run the Logic Simulator, please consult the \verb+README.md+. 

\includepdf[pages={1-}]{../second_interim_report/example_circuits.pdf}

\end{document}