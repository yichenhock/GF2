%% LyX 2.2.4 created this file.  For more info, see http://www.lyx.org/.
%% Do not edit unless you really know what you are doing.
\documentclass[english]{article}
\usepackage{lmodern}
\usepackage[T1]{fontenc}
\usepackage[latin9]{inputenc}
\usepackage{geometry}
\geometry{verbose,tmargin=3cm,bmargin=3cm,lmargin=2.5cm,rmargin=2.5cm}
\usepackage{textcomp}
\usepackage{amsmath, amstext}
\usepackage{graphicx, float}
\usepackage{pdfpages}
\usepackage{booktabs}
\usepackage{multirow}
\usepackage{pdfpages}
\usepackage{fancyvrb}

\usepackage{pdflscape}

\usepackage{listings,multicol}
\lstset{
basicstyle=\small\ttfamily,
columns=flexible,
breaklines=true
}

\makeatletter

%%%%%%%%%%%%%%%%%%%%%%%%%%%%%% LyX specific LaTeX commands.
%% Because html converters don't know tabularnewline
\providecommand{\tabularnewline}{\\}

\makeatother

\usepackage{babel}
\begin{document}

\begin{center}

{\bf ENGINEERING TRIPOS PART IIA}
\end{center}
\vspace{0.5cm} {\bf PROJECT GF2 \hfill SOFTWARE}
\vspace{0.5cm}
\begin{center}
{\bf LOGIC SIMULATOR \\\hfill
\\{\em First Interim Report}\\\hfill
\\Name: Yi Chen Hock, Michael Stevens, Cindy Wu
\\College: Queens', Robinson, John's\\\hfill
\\
}
\end{center}
\rule{15.7cm}{0.5mm}

\vspace{0.5cm}
\tableofcontents

\newpage
\section{Introduction}
%-------INTRODUCTION WRITE HERE-------
The aim of this project is to develop a logic circuit simulator in Python. As a team of three, we will work together on the project under a real-life simulation of a professional software development environment. This will involve the five major phases of the software engineering life cycle: specification, design, implementation, testing and maintenance. This report will cover the general approach of the project including teamwork planning; EBNF for syntax; identification of all possible semantic errors; description of error handling; and two example definition files together with diagrams showing the logic circuits they represent. 

\section{General Approach}
%-------GENERAL APPROACH WRITE HERE-------
The general approach to the project is to split it into four main sections. The first section involves defining a precise specification for the logic description language, which the client will use to define their logic circuit. The aim of this is to create a simple and coherent language using EBNF notation, which is readable to both the user as well as the computer. In addition, the specification will identify the semantic constraints which apply to the language as well as consider how any error conditions will be handled. The details of this section are outlined in this report.

\noindent\\
The second section involves developing the individual modules for the bulk of the logic simulator. The scanner and parse modules will directly use the logic description language as well as the specification of error handling as defined in the first section. The approach for this is to create a detailed specification in the first section to allow for easier and efficient development of the code. The user interface will be designed to be straight forward as well as intuitive to use. Throughout this section, unit tests will be implemented to ensure that each module works as expected, even when given unexpected inputs. 

\noindent\\
The third section is to integrate the various modules together and ensure that the program works as one. The unit tests for each module developed in the previous section will greatly improve the speed and efficiency of integration as they will ensure that any bugs found will be confined to the integration of the modules rather than the modules themselves. Therefore, it is vital that well written and complete unit tests are developed to allow for smooth integration. In addition, clear and easy to read documentation will be produced for the client.

\noindent\\
Finally, the last section is involved with implementing any modifications to the system which the client proposes. These modifications will not be known until after the logic simulator is developed and hence, the code will be structured with plenty of modularisation. This will allow system modifications to be implemented more efficiently and with less bugs.

\subsection{Teamwork Planning}
%-------TEAMWORK PLANNING WRITE HERE-------
The Gantt chart in the Appendix (Figure \ref{fig:Gantt Chart}) shows how the tasks for this project have been split up and allocated to each team member. Note that the allocated person(s) for each task will not necessarily be the only person working on that task since the workload for each task may not be predictable at this stage of the project. In particular, the development of unit tests for each module will likely be developed by a different team member than the module's author, however, a single member will be responsible for the overall completion of these unit tests. The Gantt chart also highlights the deadlines given by the client. It's also to be noted that the aim is to complete the first iteration of the code within the first two weeks of the project, to allow sufficient time for integration and testing of the system. Throughout this project, git is used for collaborative coding and version control. The repository can be found at https://github.com/yichenhock/GF2.


\section{EBNF for Syntax}
%-------EBNF WRITE HERE-------
The logic description language will be specified using EBNF notation. The language was designed with ease of readability in mind, so English keywords and puncuation have been used such that the meaning is clear to a human reader as well as a machine. There are four main sections to the description file: 
\begin{enumerate}
    \item \verb+device+: Defines the device and device type by user-defined names and along with the number of input ports for each device and defines any switches or clocks. 
    \item \verb+initialise+: Sets the state of the switches (which will be the inputs to the logic circuit) and the frequency of the clock. Error handling will be implemented to ensure that 
    \item \verb+connections+: Defines the connections between switches, devices and input ports. This is subdivided into sections which contain all inputs to a given device.
    \item \verb+monitors+: Defines the output signals which  will be monitored. 
\end{enumerate}

\noindent The following is a formal description using EBNF of the logic description language we will be using in this project. Note the top-down structure starts from the overall file format and finishes at the definition of allowed letters, symbols and numbers. 

The gate and dtype input and output names are defined by the specification set by the customer. The clock is defined by a cycle length, which is the number of simulation cycles for which the output state remains constant. The number of inputs and outputs for different gate types are not defined in the EBNF, instead being classed as semantic errors. The switch output is set by switch\_level. 

Since devices can be named with any combination of letters and numbers, subject to the condition that the start must be a letter, we have chosen lowercase letters only for 'letter', reserving capital letters for fixed keywords and gate types. This avoids common naming clashes.

In the choosing of the tradeoff between flexibility and complexity, we have given three options for defining connections (arrow, 'is connected to' and 'to') which can be ideally chosen for depending on the client's preference, but kept consistent in usage throughout any one file. The same applies for switch levels (1/0 or 'high/low').

\begin{lstlisting}
    circuit_description = device_block, initialise_block, connections_block, [monitors_block];

    device_block = "device", bracket_open, {gate_definition | dtype_definition | switch_definition | clock_definition}, bracket_close, semicolon;

    initialise_block = "initialise", bracket_open, {switch_initialisation | clock_initialisation}, bracket_close, semicolon;

    connections_block = "connections", bracket_open, {device_name, bracket_open, 
    {connection_definition}, bracket_close, semicolon}, bracket_close, semicolon;

    monitors_block = "monitors", bracket_open, {device_name | dtype_output_name | switch_name | clock_name}, bracket_close, semicolon;

    gate_definition = device_name, {[comma, device_name]}, definition, gate_type, semicolon;

    dtype_definition = device_name, {[comma, device_name]}, definition, ("DTYPE" | "DTYPES"), semicolon;

    switch_definition = switch_name, {[comma, switch_name]}, definition, ("SWITCHES" | "SWITCH"), semicolon;

    clock_definition = clock_name, {[comma, clock_name]}, definition, ("CLOCK" | "CLOCKS"), semicolon;

    gate_inputs = device_name, {[comma, device_name]}, possession, {digit}, ("input" | "inputs"), semicolon;

    switch_initialisation = switch_name, {[comma, switch_name]}, definition, switch_level, semicolon;
    
    clock_period = clock_name, ("cycle length" | "cycle"), {digit}, semicolon;

    connection_definition = (device_name | dtype_output_name | switch_name | clock_name), ("is connected to" | "to" | "connects to" | arrow), {(gate_input_name | dtype_input_name)}, semicolon;

    gate_input_name = device_name, dot, "I", {digit};

    dtype_input_name = device_name, dot, dtype_inputs;

    dtype_output_name = device_name, dot, dtype_outputs;

    device_name = letter, {[letter|digit]};

    clock_name = "clk", {[digit]};

    switch_name = "sw", {[digit]};

    gate_type = "AND" | "ANDS" | "OR" | "ORS" |"NOR" | "NORS" | "XOR" | 
                "XORS" | "NAND" | "NANDS";

    switch_level = "HIGH" | "LOW" | "1" | "0";

    dtype_inputs = "DATA" | "CLK" | "SET" | "CLEAR";

    dtype_outputs = "Q" | "QBAR";

    letter = "a" | "b" | "c" | "d" | "e" | "f" | "g" | "h" | "i" |
              "j" | "k" | "l" | "m" | "n" | "o" | "p" | "q" | "r" |
              "s" | "t" | "u" | "v" | "w" | "x" | "y" | "z" ;

    digit = "0" | "1" | "2" | "3" | "4" | "5" | "6" | "7" | "8" | "9" ;

    definition = "is" | "are";

    possession = "has", "have";

    arrow = "->";

    bracket_open = "(";

    bracket_close = ")";

    semicolon = ";";

    comma = ",";

    dot = ".";
\end{lstlisting}

\vspace{0.1in}

\noindent 
We do not include comment syntax in EBNF as the scanner removes comments from input file. The syntax for comments will be "\verb+# ;+", where "\#" denotes the start of the comment section, and ";" denotes the end of the comment section. There will be no line comments as line breaks and spaces hold no significance. Additional notes on the ENBF syntax include: 
\begin{itemize}
    \item AND, OR, NAND and NOR gates can have from 1 to 16 inputs but XOR gates can only have 2 inputs. 
    \item The monitor block is optional. The user may define it later or they may choose to not monitor any outputs. 
    \item When defining the circuit, there are a couple of phrase options which may be used. For example, when specifying connections, both "is connected to" and "to" can be used, as shown in Example A and B respectively".  
    \item The clock cycle length $n$ is number of simulation cycles for clock to change. 
    \item The DTYPE bistable has two outputs, Q and QBAR, where QBAR is the inverse of Q. 
\end{itemize}

\subsection{Example A --- Simple Circuit}
%-------SIMPLE CIRCUIT WRITE HERE-------

\vspace{-0.4cm}
\begin{figure}[H]
    \begin{center}
    \includegraphics[width = 0.7\textwidth]{Simple Circuit.png}
    \vspace{-1cm}
    \caption{XOR Gate}
    \end{center}
    \label{fig:Simple Circuit}
    \end{figure}

\begin{multicols}{2}
\begin{Verbatim}[numbers=left,xleftmargin=5mm]
    devices(
        a is OR;
        b is NAND;
        c is AND;
        a, b, c have 2 inputs;
        sw1, sw2 are SWITCHES;
    );
    
    initialise(
        sw1, sw2 are HIGH;
    );
    
    connections(
        a(
            sw1 is connected to a.I1;
            sw2 is connected to a.I2;
        );
    
        b(  
            sw1 is connected to b.I1;
            sw2 is connected to b.I2;
        );
    
        c(  
            a is connected to c.I1;
            b is connected to c.I2;
        );
    );
    
    monitors(
        c;
    );
\end{Verbatim}
\end{multicols}
    
\subsection{Example B --- Complex Circuit}
%-------COMPLEX CIRCUIT WRITE HERE-------
    
\vspace{-0.4cm}
\begin{figure}[H]
\includegraphics[width = \textwidth]{Complex Circuit.png}
\vspace{-1cm}
\caption{2 Bit Counter with Outputs XOR}
\label{fig:Complex Circuit}
\end{figure}

\begin{multicols}{2}
\begin{Verbatim}[numbers=left,xleftmargin=5mm]
    devices(
        a is NOR;
        b, c, d, e are NANDS;
        f is DTYPE;
        g is XOR;
        a, b, c, d, e, g have 2 inputs;
        sw1, sw2 are SWITCHES;
        clk1 is CLOCK;
    );
    
    initialise(
        sw1, sw2 are LOW;
        clk1 cycle length 1000;
    );
    
    connections(
        a(
            e to a.I1;
            e to a.I2;
        );
    
        b(  
            e to b.I1;
            clk1 to b.I2;
        );
    
        c(  
            clk1 to c.I1;
            a to c.I2;
        );
    
        d(
            b to d.I1;
            e to d.I2;
        );
    
        e(  
            d to e.I1;
            c to e.I2;
        );
    
        f(  
            f.QBAR to f.DATA;
            d to f.CLK;
            sw1 to f.SET;
            sw2 to f.CLEAR;
        );
    
        g(  
            d to g.I1;
            f.Q to g.I2;
        );
    );
    
    monitors(
        d, f.Q, g;
    );
\end{Verbatim}
\end{multicols}

\section{Errors}
\subsection{Error Handling}
All errors will be reported with error, function and line. A cursor on a printed current input line shows exactly at which point failure occurs, using a function in the scanner class. A counter of number of errors will be included in an error function. Each line in the specification should be ended by a semicolon, so the parser skips to past the next semicolon and resumes reading from there.

Syntax errors involve errors which do not conform to the format specified by the EBNF grammar.

\subsection{Semantic Error Identification}

The below table summarises types of semantic error, detection and reporting.

\begin{table*}[ht]
\centering
\caption{Table of semantic errors.\label{table:rotor}}
\vspace{\baselineskip}

\begin{tabular}{p{0.4\linewidth}p{0.3\linewidth}p{0.2\linewidth}}
\hline
Semantic constraint & Example of error & Reporting error message\\ [0.1ex]
\hline\hline

Clock and switch must have 1 output, 0 inputs &  $\pm$ \\ 
\hline
AND, NAND, and NOR must have 1 output, 1-16 inputs & &\\
\hline
XOR must have 1 output, 2 inputs & 32 $\pm$ 1 & -\\
\hline
Leading edge angle $\Xi_1$ & - \\
\hline
Trailing edge angle $\Xi_2$ & - \\
\hline

\end{tabular}
\end{table*}

\begin{landscape}
\section{Appendix}
%-------APPENDIX WRITE HERE-------
% \vspace{-1cm}
\begin{figure}[H]
\begin{center}
\includegraphics[width = \paperwidth]{Gantt Chart.pdf}
%\includegraphics[width = \paperwidth]{gantt.png}
\vspace{-3cm}
\caption{Gantt Chart}
\label{fig:Gantt Chart}

\end{center}
\end{figure}
\end{landscape}

\end{document}
